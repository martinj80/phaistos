\section{CRISP local move (\texttt{crisp})}

CRISP (Concerted Rotations Involving Self-consistent Proposals) is a
Monte Carlo move that produces deformations in a small segment of the
protein backbone, keeping the positions of all atoms outside the
segment fixed.  Compared to other local move approaches
\cite{betancourt2005efficient, ulmschneider2003monte}, CRISP makes it
possible to generate subtle and uniform tentative updates without
disrupting the local geometry of the chain.  For a complete
description of the method, please refer to our recent paper on the
move \cite{bottaro2012crisp}.

\optiontitle{Settings}
\begin{optiontable}
  \option{move-length-min}{int}{5}{(see below)}
  \option{move-length-max}{int}{5}{By default a CRISP move modifies
    all the bond and torsion backbone angles of 5 consecutive residues
    at a time.  The number of residues involved in a single update is
    randomly chosen in the interval [\texttt{move-length-min},\texttt{move-length-max}].}
  \option{std-dev-phi-psi}{real (degrees)}{5}{Parameter controlling
    the amplitude (in degrees) of $\phi$ and $\psi$ angles variation. Large values
    correspond to large changes.  The default value of this setting, together with the default value of 
    \texttt{std-dev-bond-angle} and \texttt{std-dev-omega} are optimized for using OPLSaa potential in
    combination with GB/SA solvation energy.}
  \option{std-dev-bond-angle}{real (degrees)}{0.8}{Parameter
    controlling the amplitude (in degrees) of bond angles variation. Large values
    correspond to large changes.}  
  \option{std-dev-omega}{real (degrees)}{0.8}{Parameter controlling
    the amplitude (in degrees) of omega angles variation. Large values correspond
    to large changes.}
  \option{only-internal-moves}{bool}{false}{Whether to perform only
    internal moves or both internal and end-moves. End-moves consist
    of small variations of bond- and torsion angles.}
  \option{sample-omega}{bool}{true}{Whether to modify omega angles,
    in addition to $\phi$,$\psi$ and bond angles during prerotation.}
\end{optiontable}

CRISP can also be used in combination with the TorusDBN model, using the
move-crisp-dbn-eh version.  In this case, a standard CRISP move is
performed. Upon the update, the likelihood of the new structure is
calculated according to the TorusDBN model (for torsion angles) and to
the Engh-Huber values (for bond-angles) \cite{engh1991accurate}.  The
approach is equivalent to proposing tentative structures according to
the prior distribution governing bond and torsion degrees of freedom.

\notification{IMPORTANT: The dbn-eh version of this move introduces an implicit energy to the simulation.}

\optiontitle{Settings specific for dbn-eh version}
\begin{optiontable}
  \option{consistency-window-size}{int}{10}{Size of window
    used (to each side) when bringing the dbn back to consistency.  A
    good value for the window size is $>7$, and a negative window size
    means that the full hidden node sequence is resampled.}
  \option{bias-window-size}{int}{10}{Size of window used when
    calculating bias. Approximates the move bias as
    $P(X[i-w,j+w])/P(X'[i-w,j+w])$, where $w$ in the window size and $[i,j]$
    is the interval where angles have been changed. A good value for
    the window size is $w>7$, and a negative window size means that the
    full bias is used.}
\end{optiontable}


\section{Semi-local move (\texttt{semilocal})}
\label{sec:semi-local-move}

The BGS (Biased Gaussian Step) move is a semi-local move in which all
the angular degrees of freedom in a small stretch of the chain are
modified \cite{favrin2001monte}.  The atomic positions from the
move-end to the end of the chain are rigidly updated.  In the BGS move,
tentative updates are drawn from a Gaussian distribution that favors
approximately local deformations of the chain.

\optiontitle{Settings}
\begin{optiontable}
  \option{move-length-min}{int}{4}{(see below)}
  \option{move-length-max}{int}{4}{By default a BGS move modifies 8
    $(\phi, \psi)$ angles in 4 consecutive residues.  The number of
    residues involved in a single update is randomly chosen in the
    interval [\texttt{move-length-min},\texttt{move-length-max}].}
  \option{only-internal-moves}{bool}{false}{Whether to perform only
    internal moves or both internal and end-moves. End-moves consist
    is small variations of bond- and torsion angles}
  \option{sample-omega}{bool}{true}{Whether to modify omega angles, in
    addition to $(\phi, \psi)$ and bond angles.}
  \option{sample-bond-angle}{bool}{false}{Whether to modify backbone
    bond angles, in addition to $(\phi, \psi)$ torsion angles.}
  \option{sample-constraint-a}{real}{300}{Global parameter controlling
    the size of the update. Large values correspond to small
    variations.}    
  \option{sample-constraint-b}{real}{10}{Parameter controlling the
    locality constraint. Large values correspond to a heavy constraint
    on the end-point displacement.}
  \option{omega-scaling}{real}{8}{If sample-omega is true, the allowed
    omega angles variation is scaled down by this factor compared to
    $(\phi, \psi)$ angles.}
  \option{bond-angle-scaling}{real}{8}{If sample-bond-angle is true, the
    allowed bond-angles variation is scaled down by this factor
    compared to $(\phi, \psi)$ angles.}
\end{optiontable}

This move also has an \texttt{dbn-eh} version, which works in the same
way as explained for the CRISP move.

\notification{IMPORTANT: The dbn-eh version of this move introduces an implicit energy to the simulation.}


\section{CRA local move (\texttt{cra})}

CRA (Concerted Rotations with Angles) is a Monte Carlo move that
produces deformations in a small segment of the protein backbone,
keeping the positions of all atoms outside the segment fixed.  The
method is constructed around the same idea as the BGS move: limiting
the movement of the endpoint of the pre-rotation, in order to increase
the probability of finding a solution for the post-rotation
\cite{ulmschneider2003monte}.


\optiontitle{Settings}
\begin{optiontable}
  \option{move-length-min}{int}{5}{(see below)}
  \option{move-length-max}{int}{5}{By default a CRA move modifies
    all the bond and torsion backbone angles of 5 consecutive residues
    at a time.  The number of residues involved in a single update is
    randomly chosen in the interval [\texttt{move-length-min},\texttt{move-length-max}]}
  \option{implicit-energy}{bool}{false}{Whether to include an implicit energy for the bond angles.
If set to true and no other moves that modifies bond angles are used, the bond angle term can be omitted in the energy function.}
% (in case bond angle term is present in the energy function)}
\end{optiontable}
