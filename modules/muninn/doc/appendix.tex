\chapter{Muninn module (\texttt{muninn})}
\label{cha:muninn-module}

This module is an interface to the Muninn generalized ensemble
framework. For details, we refer to the main publication \cite{frellsen12}.
% , and the user manual available at \url{}.

\section{Scripts}
\label{sec:muninn-scripts}

The module includes a number of Phaistos-specific Muninn scripts,
facilitating the analysis of simulations conducted through Muninn.

\subsection{1D plot (\texttt{plot\_1d.py})}
\label{sec:1d-plot}

This script allows you to plot the free energy as a function of a
given reaction coordinate. The script takes its input in a column
format, where the relevant columns are specified as parameters. By
default, the program assumes that column two holds the total energy,
and column three the reaction coordinate. This is compatible with the
observable output in Phaistos (where column one contains an ID).

\optiontitle{Command line options}
\begin{optiontable}
  \option{muninn-log-file}{string}{}{Path to Muninn log file.}
  \option{observable-file}{string}{}{Path to data file in column format.}
  \option{column-index-energy}{int}{1}{Column index at which energies are found (zero-based indexing).}
  \option{column-index-x}{int}{2}{Column index at which x-values are found (zero-based indexing).}
\end{optiontable}


\subsection{2D plot (\texttt{plot\_2d.py})}
\label{sec:2d-plot}

This script is similar to the 1D case, but creates a color-coded 2D
plot of the free energy as function of two reaction coordinates. As
for the 1D case, input is assumed to be in column format, and the
default columns used are column two (energy), three (reaction
coordinate 1) and four (reaction coordinate 2).

\optiontitle{Command line options}
\begin{optiontable}
  \option{muninn-log-file}{string}{}{Path to Muninn log file.}
  \option{observable-file}{string}{}{Path to data file in column format.}
  \option{column-index-energy}{int}{1}{Column index at which energies are found (zero-based indexing).}
  \option{column-index-x}{int}{1}{Column index at which x-values are found (zero-based indexing).}
  \option{column-index-y}{int}{2}{Column index at which y-values are found (zero-based indexing).}
\end{optiontable}


\subsection{Assign weights (\texttt{assign\_weights.py})}
\label{sec:assign-weights}

For further processing, it can be convenient to calculate the Muninn
weights for a set of PDB files. This script will return a list of (ID,
weight) pairs given an input file containing IDs and energies. By
default the input format is assumed to have IDs in the first column
and energy in the second.

\optiontitle{Command line options}
\begin{optiontable}
  \option{muninn-log-file}{string}{}{Path to Muninn log file.}
  \option{observable-file}{string}{}{Path to data file in column format.}
  \option{column-index-id}{int}{0}{Column index at which IDs are found (zero-based indexing).}
  \option{column-index-energy}{int}{1}{Column index at which energies are found (zero-based indexing).}
\end{optiontable}


% \subsection{Reweight PDB ensemble (\texttt{reweight\_pdb\_ensemble.py})}
% \label{sec:reweight-ensemble}

% This script allows you to draw canonical samples from a directory of PDB
% structures sampled from a generalized ensemble. The energies of the
% original samples must be specified either as part of their filename,
% or as part of an column input data file (as described above). Note
% that this script only produces accurate results if you generate
% significantly fewer samples than the size of your original PDB data
% set.

% \optiontitle{Command line options}
% \begin{optiontable}
%   \option{muninn-log-file}{string}{}{Path to Muninn log file.}
%   \option{pdb-directory}{string}{}{Directory in which PDB files are found.}
%   \option{observable-file}{string}{}{Optional observable file containing energies for each PDB file.}
%   \option{iterations}{int}{100}{Number of generated samples - note: this should be significantly smaller than the number of files in pdb-directory.}
% \end{optiontable}


