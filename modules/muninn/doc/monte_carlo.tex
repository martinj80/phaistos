\section{Generalized Ensembles: Muninn (\texttt{muninn})}
\label{sec:muninn}


The Metropolis-Hastings method sometimes suffers from long convergence
time. This is partly due to the fact that the forcefields used for
simulations correspond to very rugged energy landscapes. The
simulation might spend a large part of its time on irrelevant local
minima. Frequently, simulations get stuck for such extended periods that it
is questionable whether ergodicity is obtained.

A potential solution to this problem is to sample from a different
distribution with fewer ergodicity problems while keeping track of
enough information to make it possible to reweight the produced samples to
correspond to samples from the original target distribution.

Muninn is a framework to conduct these types of simulations. In the
default case, Muninn will sample from the \emph{multicanonical}
distribution \cite{berg1992multicanonical}, which corresponds to a uniform distribution over
energies. The method approximates the density of states using
histograms over energy bins. Using the density of states, the ensemble
produced by a Muninn simulation can be transformed back into a
canonical ensemble at any specified temperature. Phaistos includes a
number of scripts to facilitate this reweigthing (See chapter
\ref{cha:muninn-module}).

There are a substantial number of settings in Muninn. In most cases,
the default values will produce good results. The most important
settings are probably the \texttt{min-beta} and \texttt{max-beta},
which will set a limit to the range of temperatures explored by Muninn
during the simulation.

It should be emphasized that in Phaistos, the Muninn beta values are
unitless. This means that the Muninn beta value $\beta_{\mathrm{m}}$
corresponds to the target distribution

\begin{equation}
P_{\beta_{\mathrm{m}}}(x) \propto \exp (-\beta_{\mathrm{m}} E_{\phi}(x) )\ ,
\end{equation}

\noindent where $E_{\phi}(x)$ is the Phaistos energy for the structure
$x$ as describe in section \ref{sec:energy_convention}. For a physical
energy function the Phaistos energy is $E_{\phi}(x) \propto
\beta_{\mathrm{p}} E(x)$, where $E(x)$ is the physical energy for the
structure $x$, $\beta_{\mathrm{p}} = (kT)^{-1}$ is the thermodynamic
beta (specified by the \texttt{weight} option for the energy) and the
temperature $T$ always is a fixed value. This means that the
distribution can be expressed as

\begin{equation}
P_{\beta_{\mathrm{m}}}(x) \propto \exp (-\beta_{\mathrm{m}}\beta_{\mathrm{p}} E_{\phi}(x) )\ .
\end{equation}

\noindent Clearly, this expression is just a normal Boltzmann
distribution $P_{\beta'} = \exp(-\beta' E_{\phi}(x))$ with $\beta' =
(kT')^{-1} = \beta_{\mathrm{m}}\beta_{\mathrm{p}}$. This means that
$P_{\beta_{\mathrm{m}}}$ is equal to the Boltzmann distribution
$P_{\beta'}$ where $T' = \frac{T}{\beta_{\mathrm{m}}}$, and
consequently we have the relation

\begin{equation}
\beta_{\mathrm{m}} = \frac{T}{T'}  \ .
\end{equation}

\noindent As an example, assume that we want to explorer the
temperature range between $273\,\mathrm{K}$ and $1000\,\mathrm{K}$. If
the weight for the physical energy function is set a weight
corresponding to $T = 300\,\mathrm{K}$ then \texttt{min-beta} and
\texttt{max-beta} should be set as
\begin{align}
\texttt{max-beta} &= \frac{300\,\mathrm{K}}{273\,\mathrm{K}} \approx 1.09890 \\[1em]
\texttt{min-beta} &= \frac{300\,\mathrm{K}}{1000\,\mathrm{K}} = 0.3 \ .
\end{align}

\optiontitle{Settings}
\begin{optiontable}
  \option{energy-min}{real}{-inf}{Lower bound on energy.}
  \option{energy-max}{real}{inf}{Upper bound on energy.}
  \option{histograms-per-reinit}{int}{4}{Number of histogram updates between every reinitialization (zero or negative means no reinitialization will be done).}
  \option{burnin-fraction}{real}{2}{The fraction of \texttt{initial-max} used for burn-in (in each thread).}
  \option{use-energy2}{bool}{false}{Use the secondary energy as an additional energy in muninn - not used to estimate histograms.}
  \option{weight-scheme}{}{multicanonical}{Which weight scheme to use: \texttt{invk} or \texttt{multicanonical}.}
  \option{slope-factor-up}{real}{0.3}{Slope factor use for the linear extrapolation of the weights, when the weights are increasing in the direction away from the main area of support.}
  \option{slope-factor-down}{real}{3}{Slope factor use for the linear extrapolation of the weights, when the weights are decreasing in the direction away from the main area of support.}
  \option{min-beta}{real}{-inf}{The minimal beta value to be used based on thermodynamics and in the extrapolation.}
  \option{max-beta}{real}{inf}{The maximal beta value to be used based on thermodynamics.}
  \option{initial-beta}{real}{uninitialized}{The initial beta value to be used (if uninitialized it takes the same value as \texttt{min-beta}).}
  \option{resolution}{real}{0.2}{The resolution for the non-uniform binner.}
  \option{initial-width-is-max-left}{bool}{true}{Use the initial bin with as maximal bin width, when expanding to the left.}
  \option{initial-width-is-max-right}{bool}{false}{Use the initial bin with as maximal bin width, when expanding to the right.}
  \option{statistics-log-filename}{string}{muninn.txt}{The filename (including full path for the Muninn statistics logfile (default is \texttt{Muninn\_[PID].txt} and is obtained by removing this option from the config file or setting it to "").}
  \option{read-statistics-log-filename}{string}{}{The filename for reading the Muninn statistics logfile. If the value difference from
the empty string (""), this log file is read and the history is set based on the content.}
  \option{log-mode}{}{all}{The log mode of Muninn: \texttt{current} or \texttt{all}.}
  \option{read-statistics-log-filename}{string}{}{The filename for reading the Muninn statistics logfile. If the value difference from
the empty string (""), this log file is read and the history is set based on the content.}
  \option{read-fixed-weights-filename}{string}{}{The filename for reading a set of fixed weights for a given region. If the value difference from the empty string (""), this file is read and the weights are fixed in the given region.}
  \option{initial-max}{int}{5000}{Number of iterations used in first round of sampling.}
  \option{memory}{int}{40}{The number of consecutive histograms to keep in memory.}
  \option{min-count}{int}{30}{The minimal number of counts in a bin in order to have support in that bin.}
  \option{restricted-individual-support}{bool}{false}{Restrict the support of the individual histograms to only cover the support for the given histogram.}
  \option{dynamic-binning}{bool}{true}{Use dynamic binning.}
  \option{max-number-of-bins}{int}{1000000}{The maximal number of bins allowed to be used by the binner}
  \option{bin-width}{real}{0.1}{Bin width used for non-dynamic binning.}
  \option{verbose}{int}{3}{The verbose level of Muninn}
\end{optiontable}


\subsection{Setting the beta values}
As mentioned, the most important settings for Muninn are probably the
beta values: \texttt{min-beta}, \texttt{max-beta} and
\texttt{initial-beta}. Convergence of the ensemble
(\texttt{multicanonical} or \texttt{invk}) rely highly on the choice
of these settings.

The \texttt{max-beta} should in general be set to or just above the
beta value of interest, which usually is 1. The \texttt{min-beta}
value should on the other hand be less than the beta value that
corresponds to the critical temperature, since the critical
temperature is considered to be the boundary where the protein goes
from an unfolded to a folded state. However, the critical temperature
is rarely known, and in that case \texttt{min-beta} value should be
set close to zero. However, the lower \texttt{min-beta} is the more
time the system will spend in the unfolded state.

The \texttt{initial-beta} value is by default set equal to
\texttt{min-beta}. During the burnin, Muninn samples from the
distribution with $\beta_{\mathrm{m}}=\texttt{initial-beta}$. This
means that \texttt{initial-beta} should be set so that during the
burin time, the Markov chain should be able to reach a state which is
representable for this distribution.

The \texttt{min-beta} and \texttt{init-beta} values are highly system
dependent.
