\section{Implicit solvent model}

Phaistos includes an implementation of the Generalized Born/solvent
accessible surface area (GBSA) implicit solvent model
\cite{qiu1997gbsa}. It exists in two versions: a basic implementation
and an implementation that uses the Phaistos' ChainTree datastructure
\cite{lotan2004algorithm} to speed up the calculation and cache
contributions between iterations.

\subsection{GBSA standard implementation (\texttt{gbsa})}

This standard implementation is extremely computationally intensive,
and is primarily intended for debugging purposes.

\subsection{GBSA cached implementation (\texttt{gbsa-cached})}

This version has been optimized so that it exploits the ChainTree
data structure both when determining Born radii and for the subsequent
calculation of solvation energies. Due to the inherent structure of
the GBSA expression, it is difficult to obtain great speedups using
caching. However, this version is typically around a factor 10 faster
than the standard version mentioned above. The parameters below
determine how aggressively to reuse calculations between iterations.

\optiontitle{Settings}
\begin{optiontable}
  \option{maximum-deviation-cutoff}{real}{0.1}{Maximum deviation allowed in born radii in two subtrees of the chaintree before it is recalculated.}
  \option{cutoff-distance-phase1}{real}{inf}{Distance beyond which contributions are set to zero in phase1.}
  \option{cached-cutoff-distance-phase2}{real}{inf}{Distance beyond which contributions are set to zero in phase2.}
\end{optiontable}



\section{OPLS all-atom energy}

This implementation of the OPLS-AA/L all atom force field
\cite{jorgensen1996development} is based on the TINKER molecular
modeling suite version 4.2 \cite{ponder2010}.

The OPLS-AA/L energy is composed of 6 terms: A Lennard-Jones potential
term (attractive van der Waals forces plus steric repulsion), a
partial charge interaction term, a torsional angle energy
term, a bond angle bending energy term, a bond stretching energy
term and a term for out-of-plane bending of trivalent atoms (improper
torsion). The non-bonded terms have been merged into a single loop for optimized
performance. This non-bonded term has furthermore been implemented in a
cached version that uses the Phaistos ChainTree to rapidly locate the
parts of the chain that needs updating (see section \ref{sec:opls-non-bonded}).

To include all the OPLS and GBSA energy terms in a simulation, you can
use the short-hand command line options:

\begin{flushleft}
\texttt{--energy opls}\hspace{1ex} or\hspace{1ex} \texttt{--energy opls-cached}
\end{flushleft}

\subsection{Torsional angle energy  \\(\texttt{opls-torsion | opls-torsion-cached})}

This term calculates the energy contribution from the local backbone
configuration (dihedral angles). The cached version only recalculates
the segment that has been resampled in a given iteration.


\subsection{Angle bend energy  \\(\texttt{opls-angle-bend | opls-angle-bend-cached})}

This term calculates the contribution from bond angles
fluctuations. The cached version only recalculates the segment that
has been resampled in a given iteration.

\subsection{Bond stretch energy \\(\texttt{opls-bond-stretch})}

This term evaluates the deviation of bond lengths from their ideal
values. Note that this is rarely a degree of freedom in Monte Carlo
moves.

\subsection{Improper torsion energy \\(\texttt{opls-improper-torsion})}

This term implements an expression for out-of-plane bending of
trivalent atoms. Note that this is rarely a degree of freedom in Monte
Carlo moves.

\subsection{Partial charge interaction energy \\(\texttt{opls-charge | opls-charge-cached})}

This term evaluates partial charge interactions. The cached version
uses the ChainTree to cache previously calculated contributions.

\optiontitle{Settings (for the cached version)}
\begin{optiontable}
  \option{cutoff-distance}{real}{inf}{Distance beyond which contributions are set to zero.}
\end{optiontable}

\subsection{van der Waals interaction energy \\(\texttt{opls-vdw | opls-vdw-cached})}

Van der Waals interactions (Lennard-Jones potential) are evaluated by
this term. The cached version uses the ChainTree to cache previously
calculated contributions.

\optiontitle{Settings (for the cached version)}
\begin{optiontable}
  \option{cutoff-distance}{real}{inf}{Distance beyond which contributions are set to zero.}
\end{optiontable}

\subsection{Non-bonded interaction energy \\(\texttt{opls-non-bonded | opls-non-bonded-cached})}
\label{sec:opls-non-bonded}

This term gathers all non-bonded OPLS terms and the GBSA solvent
calculation into a single term, thereby speeding up calculations.  The
cached version of this term is the recommended way to run OPLS+GBSA
simulations. 

\optiontitle{Settings (for the cached version)}
\begin{optiontable}
  \option{vdw-cutoff-distance}{real}{inf}{Distance beyond which vdw contributions are set to zero.}
  \option{charge-cutoff-distance}{real}{inf}{Distance beyond which vdw contributions are set to zero.}
  \option{gbsa-maximum-deviation-cutoff}{real}{0.1}{Maximum deviation allowed in born radii in two subtrees of the chaintree.}
  \option{gbsa-cutoff-distance-phase1}{real}{inf}{Distance beyond which gbsa contributions are set to zero in phase1.}
  \option{gbsa-cutoff-distance-phase1}{real}{inf}{Distance beyond which gbsa contributions are set to zero in phase2.}
\end{optiontable}


