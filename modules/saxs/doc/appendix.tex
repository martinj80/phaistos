\chapter{SAXS module (\texttt{saxs})} \label{cha:saxs-module}

Small angle X-ray scattering (SAXS) is an experimental technique which
can readily provide low resolution structural information on proteins
in solution. The experiment requires relatively low amount of sample:
typical quantities are a volume of $\sim15\,\mu l$, and a sample
concentration of $\sim$1.0\, mg/ml~\cite{SvergFT-1987}.

SAXS was originally developed for structure analysis of condensed
matter in isotropic environments and the first applications date back
to the 1930s, with the publication of studies on metal
alloys~\cite{Guini-1939}. The method was later extended to other
materials, included biological macromolecules in solution~\cite{KratkG-1982}.

The information that can be obtained directly from a SAXS experiment
includes the radius of gyration, volume, molecular mass (provided a
known reference is available) and possibly a coarse-grained
description of the electron distribution in the sample. These
parameters are inferred from inverse Fourier analysis of the SAXS
experiment, and are limited by the noise level of the
instrumentation. In particular, the determination of the electronic
distribution function $p\left(r\right)$ requires a number of
regularization steps whose relative weight is unclear. Manual tuning
is therefore typically required to produce satisfactory
results~\cite{SemenS-1991}.

Since the 1980s, improvement in the instrumentation has led to
reliable analyses of polymeric systems, and increase in computation
power during the 1990s allowed for Bayesian regularization of the
inverse Fourier analysis, providing better estimators and more
rigorous error analysis~\cite{Hanse-1990}.  The latter is
unfortunately still not routinely applied.

Historically, analysis of obtained SAXS curves was limited to a
comparison to scattering curves describing simple geometrical bodies
with known analytical solutions. As sufficient computational power
became available, this analysis was extended with the possibility of
evaluating forward models of SAXS data from a hypothetical
experiment. These structural proposals are typically compared to the
experimental result in order to validate one or more structural
models~\cite{KochVS-2003,KratkG-1982,SvergFT-1987}.

There are two main approaches for calculating a theoretical SAXS curve
from the structure (the forward model). One is based the approximation
of the electronic inhomogeneity via spherical harmonics
expansion~\cite{svergun1995crysol}, the other evaluates the pairwise
interactions occurring between elementary scatterers by means of
direct Fourier transforms~\cite{glatter1982small}.  The spherical
harmonics approach is computationally attractive, but it is also quite
sensitive to the correct determination of the atomic volume of each
individual scatterer. Further, the algorithm enforces a compact,
quasi-globular shape, and the evaluation has to be completely
recomputed even in case of a minor update in the proposal. For this
reason, Monte Carlo protocols do not fully benefit of this method.

The SAXS module in Phaistos follows the description using
pairwise interactions. This is based on the Debye formula~\cite{Debye-1915}

\begin{equation}
I(q)=\sum_{i=1}^{M}\sum_{j=1}^{M}F_{i}(q)F_{j}(q)\frac{\sin(q\cdot r_{ij})}{q\cdot r_{ij}} \ ,
\end{equation}

\noindent where the experimental SAXS curve $I(q)$ is defined as the
sum of the pairwise interactions between the $M$ bodies in the
system. These are individually described by the product of the form
factors, $F_{i}(q)$, with the structure factors depending on the
distance between them ($r_{ij}$) and the scattering momentum
$q=\frac{4\pi}{\lambda}\sin\theta$. Here, $\theta / 2$ is the
scattering angle between the incoming beams, and $\lambda$ the X-ray
wavelength. This approach is computationally more expensive than the
spherical harmonics evaluation ($O(n^2)$) but allows for partial
updates in the structure and hence caching. Importantly, it is not
limited by assumptions on the protein shape.

An attractive performance can be achieved by reducing the number of
elementary scatterers. By grouping neighboring atoms in a set of
\emph{dummy bodies}, the $I(q)$ profile can be conveniently
approximated at least to the current experimental
resolutions~\cite{stovgaard2010calculation}. The Phaistos SAXS module
includes parameters for two levels of coarse-graining. The one-body
model describes each protein residue a single scattering body, whereas
the two-body model used one (common) dummy body for the backbone part
of the residue and one for the side chain scattering. The two-body
model is significantly more accurate for resolutions above $q \approx
0.2 \mathrm{\AA}^{-1}$ while the one-body model evaluates around three
times faster.  A detailed description of the coarse-grained models can
be found in the publication~\cite{stovgaard2010calculation}.

The SAXS module is implemented as an energy term for Phaistos and the
configuration is described in Section~\ref{sec:saxs-energy}.
