
\section{SAXS energy (\texttt{saxs-debye})}
\label{sec:saxs-energy}

The SAXS energy term provides support for sampling under restraints
from Small Angle X-ray Scattering (SAXS) data. It is described in
detail in Appendix \ref{cha:saxs-module}.

\optiontitle{Settings}
\begin{optiontable}
  \option{debug}{int}{0}{Print debugging information.

    5: Basic information on loading the data bases.

    10: Preparation of the error model. Energy is printed for every evaluation.

    50: Filling the form factors for the molecule in exam. Sine lookup
    table extensive validation: computes the output of the sin()
    function, and warns the user if the absolute discrepancy with the
    lookup table is above $10^{-3}$.

    100: Full internal debug information.}

  \option{saxs-intensities-filename}{filename (mandatory)}{}{The reference
    experimental reading. It should be discretized at the same
    \emph{q} bins as the form factors database (see next option). The
    discretization is performed at $q=0.015 \mathrm{\AA}^{-1}$.}

  \option{saxs-form-factors-filename}{filename (mandatory)}{}{The form
    factors database, as available as open data from our article \cite{stovgaard2010calculation}. It should be
    discretized at the same \emph{q} bins as the experimental
    reading.}

  \option{exp-errors-alpha}{real}{0.15}{Influences the amplitude of the
    scattering error, as per ref. \cite{stovgaard2010calculation}.}

  \option{exp-errors-beta}{real}{0.30}{Influences the amplitude of the
    scattering error, as per ref. \cite{stovgaard2010calculation}.}

  \option{q-bins}{int}{51}{Sets the total number of elements read from
    the \texttt{saxs\_intensities\_filename} and
      \texttt{saxs\_form\_factors\_filename} data bases.}

  \option{q-bins-first}{int}{0}{Sets the initial element of the
    \texttt{saxs\_intensities\_filename} and \texttt{saxs\_form\_factors\_filename} data bases
    (can be used to skip the initial bins of the profile).}

  \option{one-body-model}{bool}{false}{Use the one- or two-body model per amino acid. The two-body model is slower but significantly more accurate at higher q-values (see ref. \cite{stovgaard2010calculation}).}

  \option{sine-lookup-table}{bool}{true}{Use a pre-computed sine table
    in the Debye formula. Depending on your hardware, this can improve
    the speed of the Debye computation up to 3x.}

\end{optiontable}

