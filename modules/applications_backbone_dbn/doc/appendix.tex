\chapter[BackboneDBN applications module]{BackboneDBN applications module\\(\texttt{applications\_backbone\_dbn})}
\label{cha:backbone-dbn-apps}

The \texttt{applications\_backbone\_dbn} module contains a number of
small applications that interface to the various backbone
probabilistic models available in Phaistos (see chapter
\ref{cha:probabilistic-models}). The programs follow the command line
syntax of the \texttt{phaistos.cpp} program, and models are set up
using the \texttt{backbone-dbn} option in the same way as described in chapter
\ref{cha:probabilistic-models}.

\section{Sampler (\texttt{sampler.cpp})}
\label{sec:sampler}

This program provides functionality to draw samples from a model. This
involves a forward-backtrack sampling in the range specified (using
the \texttt{start-index}, \texttt{end-index} options), and a
subsequent sampling of values for the output nodes. From the command
line, it is possible to specify which output nodes you want to sample
(\texttt{sample-aa}, \texttt{sample-ss}, and
\texttt{sample-angles}). The \texttt{iterations} option specifies how
many samples to generate.


\section{Predictor (\texttt{predictor.cpp})}
\label{sec:predictor}

The \texttt{predictor.cpp} program uses either Viterbi decoding or
Posterior decoding to \emph{predict} a sequence of output values. This
choice is made using the \texttt{mode} option. The Viterbi mode works
by finding the most probably hidden node sequence path through the
model, and then outputting the corresponding emission probabilities
for each position. Posterior decoding works by choosing the state with
maximum probability at each position in the sequence, and then
outputting the corresponding emission probabilities. Using the
\texttt{output-mode} option, you can choose to predict either the
amino acid sequence (\texttt{aa}) or the secondary structure sequence
(\texttt{ss}).


\section{Likelihood (\texttt{likelihood.cpp})}
\label{sec:likelihood}

This program simply calculates the probability of the input given the
model. You can specify the range of the sequence to evaluate
(\texttt{start-index} and \texttt{end-index}). Optionally, you can choose
to subtract the likelihood of the amino acid sequence from the total
likelihood -- this will give you the likelihood \emph{conditioned} on
the amino acid sequence.


\section{Entropy (\texttt{entropy.cpp})}
\label{sec:entropy}

The \texttt{entropy.cpp} calculates the structural entropy for a given amino
acid sequence. This can be viewed as a measure of the flexibility of a
sequence, or the entropy of the unfolded state of a sequence. This is carried
out by estimating $\langle P \ln P \rangle$ through resampling local structures
conditioned on the amino acid sequence. Note that only local structure is taken
into account, effects such as steric repulsion are not. 

\section{Check mutations (\texttt{check\_mutations.cpp})}
\label{sec:check-mutations}

The \texttt{check\_mutations.cpp} program calculates the change in log
likelihood of the local structure for every possible single mutation of a
protein structure. In other words, the changes reflect whether a
given point mutation is compatible with the local structure of the WT structure. 


