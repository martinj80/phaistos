\chapter{GIT module (\texttt{git})}
\label{cha:git-module}


\section{PDB file to GIT vectors (\texttt{pdb2git})}
\label{sec:pdb2git-executable}

The \texttt{pdb2git} program is a convenient way to transform a set of
regular PDB files into a GIT vector file. The conversion itself only
takes fractions of a second, but due to I/O and the fact that one
often want to convert the several thousand decoys generated by a long
simulation, the conversion may take minutes.

\optiontitle{Command line options}
\begin{optiontable}
  \option{git-file}{filename}{\emph{required}}{The outputfile to store
    the generated vectors in. Lines will be appended to an existing
    file.}

  \option{verbose}{bool}{false}{Whether produce a verbose output
    (recommended if you want to see what is going on).}

  \option{debug}{bool}{false}{Whether to print debugging information.}

  \option{input-file}{filename}{}{Calculate the GIT vector for a
    specified file only.}

  \option{input-dir}{path}{}{Calculate the GIT vectors for all
    (\texttt{*.pdb|*.ent}) files in the directory \texttt{path}. The
    directory \texttt{path} should with a "/".}
\end{optiontable}
  
\subsection{Examples (cookbook style)}
\label{sec:examples-pdb2git}

The main application of the \texttt{pdb2git} is to transform entire
directories of decoy structures into GIT vectors. The list of options
for \texttt{pdb2git} can be obtained using the command:

\begin{verbatim}
$ ./pdb2git --help
\end{verbatim}

\noindent In order to transform an entire directory of decoy
structures into git vectors use the \texttt{input-dir} option. Is it
also necessary to specify a target file, where all the generated
vectors will be stored:

\begin{verbatim}
$ ./pdb2git --git-file my-git-file.git \
  --input-dir ../folding-simulation/samples/ --verbose
\end{verbatim}


\subsection{Frequently asked questions (FAQ)}
\label{sec:faq-pdb2git}

\begin{description}
\item[What is GIT?] GIT is a description of the overall protein fold
  using Gauss integrals. It was developed by R{\o}gen and Fain, and
  the distances between the vectors has been shown to correlate well
  with standard measures like RMSD \cite{roegen2003automatic}.
\end{description}

