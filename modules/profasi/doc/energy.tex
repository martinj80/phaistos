
\section{PROFASI energy}
\label{sec:profasi-energy}

The PROFASI force field implemented in Phaistos is an implicit solvent
all-atom potential for simulations of protein folding and
aggregation. The potential is developed through studies of structural
and thermodynamic properties of 17 peptides with diverse secondary
structure \cite{irback2009effective}. It is composed of four terms

\begin{equation}
E = E_{\mathrm{loc}} + E_{\mathrm{ev}} + E_{\mathrm{hb}} + E_{\mathrm{sc}} \ ,
\end{equation}

\noindent
where $E_{\mathrm{loc}}$ is a local electrostatic potential between atoms
separated by a few covalent bonds, $E_{\mathrm{ev}}$ is a potential of the
excluded volume effect, $E_{\mathrm{hb}}$ is a hydrogen bond potential,
and $E_{\mathrm{sc}}$ represents interactions between sidechains. These four
terms are divided into seven parts in the implementation for the sake
of optimization. The individual terms are briefly discussed
below. There are various versions of each term. Typically, either the
\texttt{cached} or the \texttt{improved} versions are faster than the
standard implementation. Which of the two are fastest depends on the
length of the protein. To give an idea of the speeds, we include
speedup factors for a particular protein (PDB Code 1lq7). In a given
simulation scenario, the user is recommended to try both variants to
see which one is most efficient.


% For a detailed description of the energy terms see
% \cite{irback2009effective}.

% They can be obtained by using the following profasi
% energy options in the settings of the simultion of Phaistos. 

% \subsection{Individual terms}
% \label{sec:profasi-individual-terms}



\subsection{Profasi local energy \\(\texttt{profasi-local | profasi-local-cached})}
This term calculates the $E_{\mathrm{loc}}$ along the backbone,
including the Coulomb interactions between partial charged atoms on
the neighbouring peptide along the backbone, and in addition, the
repulsion of the HH atoms and the OO atoms between neighbouring
peptides.

The cached version of the profasi local energy term, uses the
ChainTree data structure, which will give the same profasi local
energy value with around 10 times faster calculation speed.

\subsection{Profasi local sidechain energy \\(\texttt{profasi-local-sidechain~|}\\\texttt{profasi-local-sidechain-cached})}
This term calculates the $E_{\mathrm{loc}}$ along the sidechains, which is the explicit
torsional angle potential.

The cached version of the profasi local sidechain energy term uses
the ChainTree data structure, which will give the same energy value
with around 6 times faster calculation speed.

\subsection{Profasi excluded volume energy \\(\texttt{profasi-excluded-volume~|}\\\texttt{profasi-excluded-volume-cached})}

This term calculates the non-local part of the $E_{\mathrm{ev}}$,
by summing over all the excluded volume potential between pairs of
atoms that are separated by non-constant disance and more than three
convalent bonds.

The cached version of the profasi non-local excluded volume energy
term uses the ChainTree data structure, which will give the same
energy value with around 15 times faster calculation speed.
 

\subsection{Profasi local excluded volume energy \\(\texttt{profasi-excluded-volume-local~|}\\\texttt{profasi-excluded-volume-local-cached})}

This term calculates the local part of the $E_{\mathrm{ev}}$ that sums
over all the excluded volume potential between pairs of atoms that are
separated by non-constant distance and three convalent bonds.

The cached version of the profasi local excluded volume energy term
uses the ChainTree data structure, which will give identical energy value to
local excluded volume with around 13 times faster calculation speed.
 

\subsection{Profasi hydrogen bond energy \\(\texttt{profasi-hydrogen-bond~|}\\\texttt{profasi-hydrogen-bond-cached~|}\\\texttt{hydrogen-bond-improved})}

This term calculates the $E_{\mathrm{hb}}$ by summing up
backbone-backbone and side\-chain-back\-bone hydrogen bonding
interactions.

The cached version of the profasi hydrogen bond energy term uses the
ChainTree data structure, which will give identical energy value to
hydrogen bond energy with around 1.8 times faster calculation speed.

The improved version of the profasi hydrogen bond energy term uses a
term-specific optimization technique, which will give identical energy
value to hydrogen bond energy term with around 4.4 times faster
calculation speed.
 
\subsection{Profasi hydrophobicity energy \\(\texttt{profasi-hydrophobicity~|}\\\texttt{profasi-hydrophobicity-cached~|}\\\texttt{profasi-hydrophobicity-improved})}
This term calculates the hydrophobic part of $E_{\mathrm{sc}}$.

The cached version of the profasi hydrophobicity energy term is using
the ChainTree method, which will give identical energy value to
hydrophobicity energy with around 1.7 times faster calculation speed.

The improved version of the profasi hydrophobicity energy term uses a
term-specific optimization method, which will give identical energy
value to hydrophobicity energy term with around 1.1 times faster
calculation speed.

\subsection{Profasi sidechain charge energy \\(\texttt{profasi-sidechain-charge~|}\\\texttt{profasi-sidechain-charge-cached~|}\\\texttt{profasi-sidechain-charge-improved})}
This term calculates the part of $E_{\mathrm{sc}}$ which is due to
sidechain charge-charge interaction.
 
The cached version of the profasi sidechain charge energy term uses
the ChainTree method, which will give identical energy value to
sidechain charge energy with around 1.5 times faster calculation
speed.

The improved version of the profasi sidechain charge energy term uses
a term-specific optimization method, which will give identical energy
value to sidechain charge energy term with around 1.4 times faster
calculation speed.




