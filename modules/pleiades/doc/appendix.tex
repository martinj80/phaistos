\chapter{PLEIADES module (\texttt{pleiades})}
\label{cha:pleiades-module}

\section{The PLEIADES program \texttt{pleiades.cpp}}
\label{sec:pleiades-executable}

PLEIADES is a protein structure clustering program that allows to quickly
cluster very large amounts of decoy structures.

\subsection{Command line options}
\begin{optiontable}
  \option{git-file}{filename}{\emph{required}}{Input file containing all the vectors to
    be clustered. See \texttt{pdb2git} if you want to convert pdb files
    into GIT vectors.}
  \option{out-file}{filename}{out.cluster}{File to store the final cluster in.}
  \option{verbose}{bool}{false}{Produce a verbose output (recommended if you
    want to see what is going on).}
  \option{debug}{bool}{false}{Print debugging information.}
  \option{iterations}{int}{100}{How many iterations to maximally run before
    slopping. Since the convergence is not always detected
    automatically you should set a reasonable amount of iterations
    after which the algorithm terminates.}
  \option{k-means}{bool}{true}{Flag indicating to perform k-means clustering.}
  \option{k}{int}{10}{Number of K cluster being used.}
  \option{w-k-means}{bool}{false}{Flag indicating to perform weighted
    k-means clustering based on weights derived from the Muninn bin
    weights. See section \ref{sec:muninn} and chapter \ref{cha:muninn-module} for details on Muninn.}
  \option{beta}{real}{1.0}{Inverse temperature for the weighted clustering.}
  \option{long-output}{bool}{false}{Prints all members of each cluster (per
    default only the 5 members closest to the centroid are printed).}
  \option{smart-seed}{bool}{false}{Use the kmeans++ algorithm to set the
    initial cluster seeds. This will choose new centroids with a
    likelihood proportional to the distance to the closest existing
    cluster centriod. This will lead to more equally distributed
    cluster seeds \cite{arthur2007k}.}
  \option{muninn-log}{filename}{}{Filepath to the muninn logfile to be
    used for the weight determination in the weighted clustering.}
  \option{scale-weights}{bool}{false}{Allows to scale the weights to the
    interval [0,1[. Use this option if you find that the weights do
    contain zeros. Zero-weights will otherwise mess up the
    clustering.}
  \option{rmsd-native-pdb}{filename}{}{It is possible
    calculated the RMSD of each structure of a cluster to a native,
    target structure after the clustering procedure. This option
    allows to set the filepath to the native structure to be used.}
  \option{rmsd-decoy-prefix}{path}{}{In order too be able
    to calculate the RMSD to the native structure for each decoy, we
    need the decoy structures themself. The filename is extracted from
    the first column of the GIT file. Please adjust this prefix, so
    that both combined points to the correct files.}
  \option{guess-k}{bool}{false}{Experimental procedure, trying to guess a
    good K by using a cutoff based clustering approach.}
  \option{guess-k-threshold}{real}{30}{Threshold for the
    experimental detection of K.}
\end{optiontable}



\subsection{Examples (cookbook style)}
\label{sec:examples}

In this section we are going to go over a few application scenarios and how those can be 
handled by the PLEIADES program.

In order to get the list for options for the \texttt{pleiades} program use the commend:

\begin{verbatim}
$ ./pleiades -h
\end{verbatim}

\noindent To then start a clustering run, you need all the structures
in GIT vector format. The GIT vectors can be obtained using the
program \texttt{pdb2git}, see section \ref{sec:pdb2git-executable}.

The following command will the start a clustering run for a maximum of
500 iterations sorting the structures into $k=20$ cluster. The option
\texttt{--long-output} indicates that all members of the clusters will
be printed and the output-file will be named \texttt{samples.cluster}.

\begin{verbatim}
$ ./pleiades -git-file ~/data/samples/samples.git --verbose -k 20 \
  --iterations 500 --long-output -o samples.cluster
\end{verbatim}


\subsection{Frequently asked questions (FAQ)}
\label{sec:faq}

\begin{description}
\item[What is GIT?] GIT is a description of the overall protein fold
  using Gauss integrals. It was developed by R{\o}gen and Fain, and
  the distances between the vectors has been shown to correlate well
  with standard measures like RMSD \cite{roegen2003automatic}.


\item[How can I calculate GIT vectors? I only have PDB files] The
  PHAISTOS package also contains a convenient conversion tool to
  transform PDB structures into the according GIT vectors called
  {\tt{pdb2git}}. See section \ref{sec:pdb2git-executable} for more
  information.

\item[How can I cite PLEIADES?] We are currently in the process of
  writing up a paper on PLEIADES.  We will update this section once
  the article is published.

\end{description}

