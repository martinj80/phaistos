\chapter{Typhon module (\texttt{typhon})}

TYPHON is a novel approach to exploring the conformational space of
proteins based on probabilistic models of protein structure.  The
incorporated models, namely TorusDBN \cite{boomsma2008gpm} and
BASILISK \cite{harder2010beyond}, ensure that the samples are
protein--like on a local length scale. In order to control the overall
structure of the protein, different types of distance based restraints
can be defined. Over the course of the simulation, the probabilistic
models will thus explore the conformational space of the protein that
is allowed by the given restraints.

There are four types of interactions currently supported in TYPHON:

\begin{description}
\item[Hydrogen Bonds] All sorts of intra-protein hydrogen bonds are
  supported and detected (using the DSSP HBond score). The bond is
  defined by four distances: N--O, N--C, H--O and H--C.  The specific
  parameters where obtain from high resolution structures.
\item[Disulfide bridges] Disulfide bridges are detected based on their
  geometry. Additionally to the sulfur-sulfur bond, the bond geometry
  is stabilized by three additional Gaussian contacts.
\item[$\mathrm{C}_\alpha$-$\mathrm{C}_\alpha$ contacts] Per default $\mathrm{C}_\alpha$-$\mathrm{C}_\alpha$
  contacts of residues that are far in the amino acid chain (5 or more
  residue) but close in space (6 {\AA}ngstrom or less) are included in
  the network. To ignore those please use the
  \texttt{--ignore-gaussians} option.
\item[Arbitrary atom-atom interaction] One can specify any atom pair
  interaction by adding it to the network file. This can for example
  be used to {\emph{pull}} certain parts of the structure closer
  together or make sure that they stay at a certain distance.
\end{description}

\section{The TYPHON program (\texttt{typhon.cpp})}
\label{sec:typhon-executable}

The TYPHON distance-restraint functionality is available as an energy
term for the Phaistos framework, described in section
\ref{sec:constr-dist-term}. For convenience, we also provide a
\texttt{typhon} executable specifically tailored for the Typhon
functionality.

\subsubsection{Common options}
\begin{optiontable}

  \option{verbose}{bool}{false}{Produce a verbose output (highly recommended if
  you want to see what is going on).}  

  \option{debug}{bool}{false}{Print debugging information.}

  \option{analyze}{bool}{false}{Analyze and print the restraint network
    only, no sampling will be performed. This options is useful when
    initially analyzing and modifying the restrain network.}

  \option{iterations}{int}{10,000,000}{For how many iterations to run the
    simulation.}

  \option{burnin}{int}{1,000}{How many iterations to skip before starting to
    sample.}

  \option{reinitialize-interval}{int}{0}{How often to
    reinitialize to the native structure, in number of iterations.}

  \option{seed}{int}{\emph{current timestamp}}{It is possible to use specific seeds for the
    random number generator. Usually you should leave this empty to
    generate a seed based on the current system time. This will ensure
    different seeds for all runs started. The seed is a natural
    number.}
\end{optiontable}

\subsubsection{Input options}
\begin{optiontable}

  \option{pdb-file}{filename}{\emph{required}}{Protein structure to start from, in the
    PDB file format.}

  \option{ss-file}{filename}{}{Optional input of the expected secondary
    structure in the Phaistos secondary structure file format, which
    in short is a one lined file with a C,E or H for each residue
    position in the structure.  Where C would be a coil conformation,
    E an extended strand and H a helical conformation.}

  \option{restore}{filename}{}{To restore a network from a
    previous run or a modified network. Expects the path to the
    network file as an argument.}
\end{optiontable}

\subsubsection{Output options}
\begin{optiontable}

  \option{output-directory}{path}{./samples/}{Where to store the generated structures. }

  \option{dump-interval}{int}{10000}{How often should a structure be saved, in number of iterations.}
  
  \option{dump-git}{bool}{false}{Activates the output in the form of GIT vectors as well.}

  \option{create-phaistos-structure}{bool}{false}{Create a "native"
    structure with all atoms at the beginning of the run. This option
    can be useful for visualization purposes and when investigating
    and/or modifying the restraint network. Note: Typhon may add atoms
    to the input structure (i.e. missing hydrogens).}

\end{optiontable}

\subsubsection{Network options}
\begin{optiontable}

  \option{no-prune}{bool}{false}{Do not prune/optimize the hydrogen bond
    network. Per default hbonds with multiple partners are removed
    along with very weak bonds due to high solvent accessibility. }

  \option{ignore-hbonds}{bool}{false}{Disregard any hydrogen bonds found
    (overwrites the restore network option, meaning that no hydrogen
    bonds will be present in the network, even if they were specified
    in the network file).}

  \option{ignore-sc-hbonds}{bool}{false}{Disregard any side chain - side
    chain hydrogen bonds found (overwrites the restore network
    option).}

  \option{ignore-bbsc-hbonds}{bool}{false}{Disregard any backbone - side
    chain hydrogen bonds found (overwrites the restore network
    option).}

  \option{ignore-ssbonds}{bool}{false}{Disregard any disulfide bridges
    found (overwrites the restore network option).}

  \option{ignore-gaussians}{bool}{false}{Disregard any Gaussian contacts
    found (overwrites the restore network option).}

  \option{init-hbond-from-native}{bool}{false}{Per default we use
    parameters estimated from the top 500 set to describe the bond
    geometry. Set this flag to initialize hydrogen bond distances from
    the native structure.}

  \option{ca-cutoff}{real}{6}{Cutoff distance in {\AA}ngstrom to be
    considered a $\mathrm{C}_\alpha$ contact.}

  \option{ca-skip}{int}{5}{How many residues to skip along the chain
    before considering a $\mathrm{C}_\alpha$ contact. Default is set to 5.  }
\end{optiontable}


\subsubsection{Path options}

If you specify the \texttt{PHAISTOS\_ROOT} environmental variable
pointing to your Phaistos build directory you should not have to worry
about these options, see section \ref{sec:locating-data-files} for
details.

\begin{optiontable}

  % \option{dbn-parameter-dir}{path}{../data/backbone\_dbn\_parameters/}{In which directory
  \option{dbn-parameter-dir}{path}{}{In which directory
    can the TorusDBN parameter set be found. }

  % \option{dbn-parameter-file}{filename}{p\_TORUS\_H55\_BIC-2070230\_FULL\_SABMARK\_NEW\_SS.txt}{Which TorusDBN
  \option{dbn-parameter-file}{filename}{}{Which TorusDBN
    parameter set to use.}

  % \option{sc-dbn-file}{filename}{../data/mocapy\_dbns/basilisk.dbn}{Which BASILISK DBN to use for the side chain movements.}
  \option{sc-dbn-file}{filename}{}{Which BASILISK DBN to use for the side chain movements.}

\end{optiontable}


\subsubsection{Move options}

\begin{optiontable}

  \option{sc-move-weight}{real}{5}{How many sidechain moves per backbone (CRISP) move. Per default we recommend 
    doing five moves, since this will amount in roughly an equal amount of motion in the 
    backbone (CRISP 5 residues) and side chains (BASILISK 1 residue). }

  % \option{use-local-sc-move}{bool}{}{Use local sidechain moves, allows
  %   for more detailed dynamics. Different move type of side chain
  %   move, where the side chain is more likely to remain in the current
  %   rotameric state. }

  %\option{local-move-weight}{real}{1}{How many local sidechain moves
  %  per CRISP move. Default is again set to five, but you may want to
  %  balance the overall sidechain movements to match the backbone
  % movement.}

\end{optiontable}


\subsubsection{Sampling options}

\begin{optiontable}
  \option{ignore-ss}{bool}{false}{Allows to sample backbone conformations
    without secondary structure input. This will generally lead to a
    broader sampling of the backbone conformational space. }
\end{optiontable}

\subsubsection{Dehydron options}

For more information and details on the dehydron concept see
reference \cite{fernandez2003dehydron}.

\begin{optiontable}
  \option{dehydron-bb-cutoff}{real}{14}{Consider backbone hydrogen bonds
    below ($d<=\texttt{cutoff}$) as dehydrated aka weak/broken. All weak hydrogen
    bonds will be removed. }

  \option{dehydron-bbsc-cutoff}{real}{9}{Consider backbone-side chain
    hydrogen bonds below ($d<=\texttt{cutoff}$) as dehydrated aka weak/broken.}

  \option{dehydron-sc-cutoff}{real}{7}{Consider side chain hydrogen
    bonds below ($d<=\texttt{cutoff}$) as dehydrated aka weak/broken.}

\end{optiontable}


\subsection{Examples (cookbook style)}
\label{sec:examples-typhon}

In this section we discuss a few application scenarios and how those
can be handled in TYPHON. We are not discussing the installation
process again as this was covered in chapter
\ref{cha:installation}. However, we would like to stress the advantage
of setting an environmental variable called \texttt{PHAISTOS\_ROOT}
pointing to your Phaistos build directory, see section
\ref{sec:locating-data-files}. With the \texttt{PHAISTOS\_ROOT} setup,
you should be able to run TYPHON and obtain the help message:

\begin{verbatim}
$ ./typhon -h
\end{verbatim}

\noindent Before starting a simulations it is probably a good idea to
inspect the network as it was detected by TYPHON. Per default TYPHON
adds all hydrogen bonds it finds to the network along with a number of
long range $\mathrm{C}_\alpha$-$\mathrm{C}_\alpha$ contacts.

\begin{verbatim}
$ ./typhon --pdb-file 2GB1.pdb --analyze --verbose \\
   --generate-pymol 
\end{verbatim}

\noindent This command will do several things. First specifying the
\texttt{--analyze} flag, we are signaling TYPHON to only calculate the
interaction network, print the result and finish without running any
simulation. The \texttt{--verbose} flag indicates, that all output is
printed and \texttt{--generate-pymol} triggers the output of a PyMOL
script visualizing the restraint network. The following command will
launch PyMOL and load the network as detected by TYPHON:

\begin{verbatim}
$ pymol load_pymol_network.py 
\end{verbatim}

\noindent From here the network can be manipulated with the expert
knowledge available through you. You know a certain bond has been
shown to be weak or especially important. You can modify the network
at will and use the \texttt{--restore} option to load the optimized
network for your simulation.

All that is really required for a dynamics simulation is a PDB file describing the protein 
structure. We do recommend to use the verbose option in order to receive a more detailed,
step by step output:

\begin{verbatim}
$ ./typhon --pdb-file 2GB1.pdb --verbose 
\end{verbatim}

Of course, in a more realistic application scenario you will want to specify more 
parameters, such as where to store the generated samples, how long to run the simulation, 
and which network to use:
 
\begin{verbatim}
$ ./typhon --pdb-file 2GB1.pdb --verbose --iterations 50000000 \
  --output-directory /my/storage/typhon/runID/samples \
  --restore optimized-network.net
\end{verbatim}


\subsection{Frequently asked questions (FAQ)}
\label{sec:faq-typhon}


\begin{description}

\item[What is a dehydron?] The concept of dehydrons is based on the
  observation that hydrogen bonds are less stable if they are exposed
  to the solvent, since water molecules will be competing for the
  charge.  Fern{\'a}dez and coworkers found an easy way to determine the
  solvent accessibility of a hydrogen bond, by counting the unpolar
  carbon atoms within a certain sphere around the bond \cite{fernandez2003dehydron}. In TYPHON, all
  weakly shielded hydrogen bonds are removed from the restraint
  network.  You can include all hydrogen bond by either decreasing the
  cutoff or using the \texttt{--no-prune} option.

\item[The program cannot find the DBN models] The DBN files are
  required to load the structural prior models or their parameters to
  be a little more specific. Those files can be found in your Phaistos
  build directory in the subfolder \texttt{./data}. You can either
  specify the paths directly (see command-line options above) or, a
  little easier, specify the \texttt{PHAISTOS\_ROOT} environmental
  variable to point to your Phaistos build directory.

\item[How can I cite TYPHON?] Harder, T., Borg, M., Boomsma, W.,
  R{\o}gen, P., Hamelryck, T. (2011) Fast large-scale clustering of
  protein structures using Gauss integrals. Bioinformatics. Accepted.

\end{description}


\section{Typhon in \texttt{phaistos.cpp}}
\label{sec:typhon-in-phaistos}

The main functional component in Typhon is an energy term called
\texttt{con\-strain-dis\-tan\-ces}. As described in chapter
\ref{sec:energies}, this energy term can be included in any
\texttt{phaistos.cpp} simulation. For convenience, the set of moves
and energies used in the \texttt{typhon.cpp} executable are grouped
together in a Phaistos mode called \texttt{typhon}. As an example, the
typhon simulation corresponding with the command

\begin{verbatim}
$ ./typhon --verbose --pdb-file 1enh.pdb --ss-file 1enh.dssp
\end{verbatim}

\noindent can be run through \texttt{phaistos.cpp} using

\begin{verbatim}
$ ./phaistos --pdb-file 1enh.pdb --init-from-pdb \
  --mode typhon --ss-file 1enh.dssp
\end{verbatim}

\noindent This makes it easy to experiment with different move sets
and energies.
