
\section{Sidechain DBN moves}
\label{sec:sidechain-move-dbn}

The sidechain DBN moves (\texttt{sidechain-basilisk} and \texttt{sidechain-compas})
allow you to sample amino acid side chain conformations from one of
the included probabilistic models of side chain geometry, namely
COMPAS or BASILISK \cite{harder2010beyond}.  These moves will propose
a new set of angles for a randomly selected stretch of residues. We
recommend only using a stretch of length one, or in other words, to
only resample a single side chain at a time. Changing multiple
sidechains in one go will increase the likelihood of introducing
clashes, leading to a significantly higher rejection rate.

Both moves use an external model file. Please remember to set the
\texttt{PHAISTOS\_ROOT} environmental variable to avoid problems
accessing these files (see section \ref{sec:locating-data-files}).

\subsection{General settings}

The behavior of the two moves is identical in terms of
parameters. However, the moves applies to different side chain
representation: BASILISK requires full atomic details, whereas COMPAS
only requires a pseudo atom at the side chain center of mass.  Both
sidechain DBN moves have the following options.

\begin{optiontable}
\option{move-length-min}{int}{1.}{Minimum move length.}
\option{move-length-max}{int}{1.}{Maximum move length.}
\option{mocapy-dbn-dir}{string}{../data/mocapy\_dbns}{Path to mocapy model file directory.}
\option{implicit-energy}{bool}{\emph{true}}{Whether the dbn bias (implicit energy) should be divided out (=false) or not (=true).}

\option{ignore-bb}{bool}{\emph{false}}{Whether to use the backbone independent version instead.}
\option{check-move}{bool}{\emph{true}}{Check that the move produced consisted atom positions.}
\option{sample-hydrogen-chis}{bool}{\emph{false}}{Sample sidechain residual dofs.}
\option{sample-hydrogen-chis-normal}{real}{1.}{Resampling constrained to a Gaussian around the current state.}
\option{sample-hydrogen-chis-sigma}{real}{0.035}{Standard deviation of Gaussian for resampling residual dofs.}  
\option{model-filename}{string}{basilisk.dbn/compas.dbn}{Model filename.}
\end{optiontable}

\notification{IMPORTANT: Both variants of the move introduce a bias (implicit energy) to the 
simulation that is possible to compensate by setting \texttt{implicit-energy}=false.}

\subsection{BASILISK move (\texttt{sidechain-basilisk})}
\label{sec:basilisk-move}

BASILISK assumes a full atom representation of the protein
sidechain. There are several variants of this move available:

\begin{flushleft}
\texttt{sidechain-basilisk}, \texttt{sidechain-basilisk-local}, and \texttt{sidechain-basilisk-multi}.
\end{flushleft}

\noindent The \texttt{local} version samples the hidden node state
based on the current angle values and subsequently samples angles
corresponding to this state, which generally has the effect of small
angle variations. The \texttt{multi} variant will do a sidechain move
in two disjunct regions of the protein chain.  See appendix
\ref{cha:sidechain-dbns-module} for details.


\subsection{COMPAS move (\texttt{sidechain-compas})}
\label{sec:compas-move}

COMPAS uses a pseudo-sidechain representation, with one atom per
sidechain. See appendix \ref{cha:sidechain-dbns-module} for details.

